\documentclass[11pt]{article}
\renewcommand{\baselinestretch}{1.05}

\usepackage{amsmath,amsthm,verbatim,amssymb,amsfonts,amscd, graphicx}
\usepackage{graphics}

\usepackage{xcolor}

\usepackage[hidelinks]{hyperref}
\usepackage{parskip}

% Code specific

\usepackage{listings}

\definecolor{mygreen}{rgb}{0,0.6,0}
\definecolor{mygray}{rgb}{0.5,0.5,0.5}
\definecolor{mymauve}{rgb}{0.58,0,0.82}

\lstset{ %
  backgroundcolor=\color{white},   % choose the background color; you must add \usepackage{color} or \usepackage{xcolor}
  basicstyle=\footnotesize,        % the size of the fonts that are used for the code
  breakatwhitespace=false,         % sets if automatic breaks should only happen at whitespace
  breaklines=true,                 % sets automatic line breaking
  captionpos=b,                    % sets the caption-position to bottom
  commentstyle=\color{mygreen},    % comment style
  deletekeywords={...},            % if you want to delete keywords from the given language
  escapeinside={\%*}{*)},          % if you want to add LaTeX within your code
  extendedchars=true,              % lets you use non-ASCII characters; for 8-bits encodings only, does not work with UTF-8
  frame=single,	                   % adds a frame around the code
  keepspaces=true,                 % keeps spaces in text, useful for keeping indentation of code (possibly needs columns=flexible)
  keywordstyle=\color{blue},       % keyword style
  language=Octave,                 % the language of the code
  otherkeywords={*,...},           % if you want to add more keywords to the set
  numbers=left,                    % where to put the line-numbers; possible values are (none, left, right)
  numbersep=5pt,                   % how far the line-numbers are from the code
  numberstyle=\tiny\color{mygray}, % the style that is used for the line-numbers
  rulecolor=\color{black},         % if not set, the frame-color may be changed on line-breaks within not-black text (e.g. comments (green here))
  showspaces=false,                % show spaces everywhere adding particular underscores; it overrides 'showstringspaces'
  showstringspaces=false,          % underline spaces within strings only
  showtabs=false,                  % show tabs within strings adding particular underscores
  stepnumber=2,                    % the step between two line-numbers. If it's 1, each line will be numbered
  stringstyle=\color{mymauve},     % string literal style
  tabsize=2,	                   % sets default tabsize to 2 spaces
  title=\lstname                   % show the filename of files included with \lstinputlisting; also try caption instead of title
}

%

\renewcommand{\contentsname}{Table des mati\`eres}

\topmargin0.0cm
\headheight0.0cm
\headsep0.0cm
\oddsidemargin0.0cm
\textheight23.0cm
\textwidth16.5cm
\footskip1.0cm

\begin{document}

\title{\textbf{Projet de Conception en Micro\'electronique Analogique} \\ R\'ealisation d'un CAN FLASH 6 bits}
\author{Mohamed Hage Hassan \\ Ferdinand Goumis}
\maketitle

\begin{abstract}

\end{abstract}

\clearpage

\tableofcontents
\clearpage

\section{Introduction}

Lorem ipsum dolor sit amet, consectetur adipiscing elit. Morbi tincidunt ut risus eget feugiat.
Etiam eget viverra purus, sed iaculis dui. Duis vel urna tempus, placerat dolor ut, molestie erat.
Praesent non dapibus tortor. Vestibulum facilisis mollis urna a fringilla. Nulla consequat lacus in dolor mattis varius.
Praesent pharetra, mauris quis pretium tincidunt, tortor risus lobortis leo, sit amet mattis metus leo a dolor.
 Curabitur mi risus, lacinia a lacus ut, porta hendrerit quam. Ut dignissim fermentum bibendum.
 Mauris auctor tincidunt enim, nec auctor mi faucibus quis. Sed volutpat augue non ligula iaculis, pharetra elementum dolor accumsan.
 Duis semper semper erat id accumsan. Cras consectetur varius varius. Sed porta semper tellus, quis auctor lacus euismod cursus.

\textit{Note :} Certaines illustrations sont extraites des projets.

\section{Cahier des charges}
Le projet n\'ecessite d'avoir :
\begin{itemize}
\item[-] Une r\'esolution du CAN-FLASH de 6 bits, ce qui implique l'utilisation de $2^{6} -1 = 63 $ comparateurs.
\item[-] Dynamique du signal en entr\'ee $V_e \in [0.5 V,\phantom{2} 2.5V]$
\item[-] Fr\'equence d'\'echantillonnage : $f_h = 20 MHz $
\item[-] \textbf{\textcolor{red}{Plus..}}
\end{itemize}

\section{Mise en place de l'\'echantillonneur-bloqueur}
Pour qu'on puisse convertir le signal, on doit impl\'ementer une \'etape d'\'echantillonnage, avec un fonctionnement \`a haute fr\'equence, avant la r\'ealisation de l'\'etape de comparaison, qui consiste \`a comparer les diff\'erents niveaux obtenus par \'echantillonnage avec l'amplitude du signal initial, modifi\'ee par le pont de diviseurs de tensions.

On utilise alors une topologie d'un \'echantilloneur-bloqueur \`a capacit\'es commut\'ees

\subsection{Principe de fonctionnement}

\begin{center}
    \textbf{Explication des diff\'erents phases de fonctionnement + \textcolor{red}{d\'emonstrations th\'eoriques}}
\end{center}

\subsection{Simulation avec des \'el\'ements id\'eaux}

\section{R\'ealisation d'un Amplificateur OTA \`a deux \'etages}

\subsection{Cahier des charges et dimensionnements}
\subsection{Simulation et optimisation}

\section{Mise en oeuvre des comparateurs synchronis\'es par horloge}

Le comparateur \`a impl\'ementer est illustr\'e par la figure X.

\begin{figure}[!htb]
\centering
%\includegraphics[scale=0.35]{comparateur_.jpg}
\caption{Comparateurs synchronis\'es par horloge avec une paire crois\'ee}
\end{figure}

\subsection{Principe de fonctionnement}
\subsection{Partie Th\'eorique}
\subsubsection{Imp\'edance diff\'erentielle n\'egative et expression du Gain}
\subsubsection{D\'etermination du courant et dimensionnements}
\subsection{Simulations/Optimisations}

\clearpage

\section{R\'ealisation du d\'ecodeur en VHDL}
\subsection{Programmation et synth\`ese automatique du circuit}
On utilise un code existant en Verilog, fait par ``Zbigniew	Jaworski'', universit\'e de Warsaw.
Ce code permet de .. $+$ \textbf{\textcolor{red}{explication}}

\begin{lstlisting}[language=Verilog]
module thermo2bin (thermob, bin)
input[62:0] thermob;
output [5:0] bin;

reg [62:0] thermo;
reg [5:0] bin, bin1, bin2;
integer i, j, k;

always @(thermob)
begin
  for (k = 0; k <=60; k=k+1)
    thermo[k] <= thermob[k] || thermob[k+1] || thermob[k+2];

  thermo[61] <= thermob[61] || thermob[62];
  thermo[62] <= thermob[62];
end

always @(thermo)
begin
  bin1 = 0;
  for(i=1; i <= 32; i=i+1)
    if (thermo[i-1] = 1'b1) bin1 = i;
end

always @(thermo)
begin
  bin2 = 0;
  for(j=1; j<=31; j=j+1)
    if(thermo[k+31] == 1'b1) bin2 = j;
end

always @(bin1 or bin2)
if (thermo[31] == 1'b1)
  bin = bin2 + 32;
else
  bin = bin1;

endmodule

\end{lstlisting}

\subsubsection{Fonctionnement}
\subsubsection{Synth\`ese}

\section{Layout}
\subsection{Mise en place des \'el\'ements du montage final}
\subsection{Simulation}

\clearpage

\addcontentsline{toc}{section}{R\'ef\'erences}

\begin{thebibliography}{9}
\bibitem{Razavi}
\textit{Design of Analog CMOS Integrated Circuits}
\\\texttt{Behzad Razavi, McGraw-Hill Higher Education }

\end{thebibliography}


\end{document}
